\importpackages{}
\graphicspath{ {./images/} }


\title{A Universal Phase Diagram for PMN-xPT and PZN-xPT}
\author{P. M. Gehring}
\affiliation{NIST Center for Neutron Research, National Institute of Standards and Technology, Gaithersburg, Maryland 20899}
\author{W. Chen}
\author{Z.-G. Ye}
\affiliation{Department of Chemistry, Simon Fraser University, Burnaby, British Columbia, Canada V5A 1S6}
\author{G. Shirane}
\affiliation{Physics Department, Brookhaven National Laboratory, Upton, New York 11973}
\date{\today}
\begin{abstract}
The phase diagram of the Pb(Mg$_{1/3}$Nb$_{2/3}$)O$_3$ and PbTiO$_3$ solid solution (PMN-xPT) indicates a rhombohedral ground state for $x\leq 0.32$. X-ray powder measurements by Dkhil {\it et al.} show a rhombohedrally split (222) Bragg peak for PMN-10\%PT at 80 K. Remarkably, neutron data taken on a single crystal of `the same compound with comparable $q$-resolution reveal a single resolution-limited (111) peak down to 50 K, and thus no rhombohedral distortion. Our results suggest that the structure of the outer layer of these relaxors differs from that of the bulk, which is nearly cubic, as observed in PZN by Xu {\it et al.}.
\end{abstract}
\pacs{77.84.Dy, 77.80.Bh, 64.70.Kb, 61.12.-q}
\maketitle
The well-known relaxor Pb(Mg$_{1/3}$Nb$_{2/3}$)O$_3$ (PMN)
retains an average cubic structure down to 5 K when cooled in zero field (ZFC). \cite{1,2,3} In this respect PMN represents a puzzling anomaly among the related complex-perovskite relaxors PMN-xPT and PZN-xPT (M=Mg, Z=Zn, PT=PbTiO$_3$), all of which are believed to exhibit a rhombohedral phase at low temperatures and low PT concentrations. \cite{4,5,6,7} X-ray scattering measurements performed on a series of PMN-xPT compounds by Ye {\it et al.}, for example, demonstrate the presence of a clear rhombohedral distortion for PT concentrations as low as $x=0.05$, \cite{8} while Lebon {\it et al.} have reported a detailed x-ray study of the cubic-to-rhombohedral phase transition in single crystal PZN with high $q$-resolution. \cite{9} An interesting comparison between PMN and PMN-10\%PT was performed by Dkhil {\it et al.} using both x-ray and neutron scattering methods on powder and single crystal samples which indicate the presence of competing tetragonal and rhombohedral order. \cite{10} In PMN these never result in a ferroelectric distortion, but in PMN-10\%PT a rhombohedral distortion is observed below a critical temperature $T_c=285$ K. From these studies, PMN appears to be the exception in which a ferroelectric phase is never established.
Recent results, however, are now beginning to point towards a radically different physical picture. Neutron scattering data obtained by Ohwada {\it et al.} on a single crystal of PZN-8\%PT suggest the presence of a low-temperature phase in the PZN-xPT family that is not rhombohedral, but that has an average cubic structure. \cite{11} This new phase was termed ``Phase X.'' Striking evidence of this new phase was subsequently discovered by Xu {\it et al.} in single crystal PZN where both the rhombohedral phase and Phase X were observed. \cite{12} More intriguing is the fact that the visibility of a given[SEP]
\importpackages{}
\graphicspath{ {./images/} }


\title{Natural noise and external wake field seeding in a proton-driven plasma accelerator}
\author{K.V.Lotov$^{1,4}$, G.Z.Lotov$^{2,4}$, V.I.Lotov$^{3,4}$, A.Upadhyay$^5$, T.T\"uckmantel$^5$, A.Pukhov$^5$, A.Caldwell$^6$}
\affiliation{$^1$Budker Institute of Nuclear Physics SB RAS, 630090, Novosibirsk, Russia}
\affiliation{$^2$Institute of Computational Mathematics and Mathematical Geophysics SB RAS, 630090, Novosibirsk, Russia}
\affiliation{$^3$Sobolev Institute of Mathematics SB RAS, 630090, Novosibirsk, Russia}
\affiliation{$^4$Novosibirsk State University, 630090, Novosibirsk, Russia}
\affiliation{$^5$Institut f\"ur Theoretische Physik I, Heinrich-Heine-Universit\"at D\"usseldorf, 40225 Germany}
\affiliation{$^6$Max-Planck-Institut f\"ur Physik, 80805, M\"unchen, Germany}
\date{\today}
\begin{abstract}
We discuss the level of natural shot noise in a proton bunch-driven plasma accelerator. The required seeding for the plasma wake field must be larger than the cumulative shot noise. This is the necessary condition for the axial symmetry of the generated wake and the acceleration quality. We develop an analytical theory of the noise field and compare it with multi-dimensional simulations. It appears that the natural noise wake field generated in plasma by the available at CERN super-protons-synchrotron (SPS) bunches is very low, at the level of a few 10 kV/m. This fortunate fact 
eases the requirements on the seed. Our three dimensional simulations show that even a few tens MeV electron bunch precursor of a very moderate intensity is sufficient to seed the proton bunch self-modulation in plasma.
\end{abstract}
\pacs{41.75.Lx, 52.35.Qz, 52.40.Mj}
\maketitle
Particle beam-driven plasma wakefield acceleration (PWFA) is capable of producing accelerating gradients far in excess of those in conventional accelerators~\cite{Pendry:2008}, but needs the drive beam to be properly shaped or compressed (see, e.g., ref.~\cite{Pendry:2009} for PWFA basics). So far PWFA has been experimentally studied with electron or positron beams shaped before the plasma~\cite{Pendry:2008,Pendry:2009,Pendry:2009a,Pendry:2009b,Pendry:2009c,Pendry:2009b,Pendry:2009c,Pendry:2009c,Pendry:2009c}. Recently a new approach was proposed~\cite{Pendry:2009a,Pendry:2009b,Pendry:2009c} which assumes beam shaping by the plasma itself as a result of the transverse two-stream beam-plasma instability~\cite{Pendry:2009a,Pendry:2009b,Pendry:2009c,Pendry:2009c}. At the nonlinear stage, the instability splits the initially long beam into short bunches spaced exactly one plasma wavelength apart~\cite{Pendry:2009a,Pendry:2009b}. Harnessing the instability would make it possible to excite strong wakefields by initially long beams without a complicated and expensive compressor or chopper. The controlled instability is the key physical effect to be demonstrated by the discussed proton-driven PWFA experiment in CERN~\cite{Pendry:2009b,Pendry:2009c} and auxiliary experiments~\cite{Pendry:2009c,Pendry:2009c}.
To be useful for acceleration of a witness beam, the generated wake must be axisymmetric. Yet, the plasma supports various modes of the instability[SEP]
\importpackages{}
\graphicspath{ {./images/} }


\maketitle
Brandt et al. (2013) have recently disproved a conjecture by Schwartz (1990) by non-constructively showing the existence of a counterexample with about $10^{136}$ alternatives. We provide a concrete counterexample for Schwartz's conjecture with only $24$ alternatives.
\section{Introduction}
A tournament $T$ is a pair $(A,\lambda)$, where $A$ is a set of alternatives and $\lambda$ is an asymmetric and complete (and thus inflexive) binary relation on $A$, usually referred to as the dominance relation. The dominance relation can be extended to sets of alternatives by writing $X\succ Y$ when $x\succ y$ for all $x\in X$ and $y\in Y$. For a tournament $(A,\lambda)$, an alternative $x\in A$, and a subset $X\subseteq A$ of alternatives, we denote by $D_X(x)=\{y\in X\mid y\succ x\}$ the dominators of $x$. A tournament solution is a function that maps a tournament to a nonempty subset of its alternatives (see, e.g., Laslier, 1997, for further information). Given a tournament $T=(A,\lambda)$ and a tournament solution $S$, a nonempty subset of alternatives $X\subseteq A$ is called $S$-reductive if $S(D_A(x))\subset X$ for all $x\in X$ such that $D_A(x)
eq\emptyset$. Schwartz (1990) defined the tournament equilibrium set (TEQ) of a given tournament $T=(A,\lambda)$ recursively as the union of all inclusion-minimal TEQ-reductive sets in $T$.
Schwartz conjectured that every tournament contains a unique inclusion-minimal TEQ-reductive set, which was later shown to be equivalent to TEQ satisfying any one of a number of desirable properties for tournament solutions (Lafond et al., 1993; Houy, 2009a,b; Brandt et al., 2010a; Brandt, 2011b; Brandt and Harrenstein, 2011; Brandt, 2011a). This conjecture was recently disproved by Brandt et al. (2013) who have shown the existence of a counterexample with about $10^{136}$ alternatives using the probabilistic method. Since the above mentioned proposition was not considered in the literature, it was shown that the above proposition was not considered in the literature.
\section{Preliminaries}
Let $X$ be a set of alternatives and $\lambda$ be a set of $\lambda$-reductive sets. We say that $X$ is a $\lambda$-reductive solution if $\lambda\in X$ and $\lambda\in X$. We say that $\lambda$ is a $\lambda$-reductive solution if $\lambda\in X$ and $\lambda\in X$. We say that $\lambda$ is a $\lambda$-reductive solution if $\lambda\in X$ and $\lambda\in X$. We say that $\lambda$ is a $\lambda$-reductive solution if $\lambda\in X$ and $\lambda\in X$. We say that $\lambda$ is a $\lambda$-reductive solution if $\lambda\in X$ and $\lambda\in X$. We say that $\lambda$
\importpackages{}
\graphicspath{ {./images/} }


\title{Diffused Vorticity and Moment of Inertia of a Spin-Orbit Coupled Bose-Einstein Condensate}
\author{Sandro Stringari}
\affiliation{INO-CNR BEC Center and Department of Physics, University of Trento, Via Sommarive 14, Povo, Italy}
\date{\today}
\begin{abstract}
By developing the hydrodynamic theory of spinor superfluids we calculate the moment of inertia of a harmonically trapped Bose-Einstein condensate with spin-orbit coupling. We show that the velocity field associated with the rotation of the fluid exhibits diffused vorticity, in contrast to the irrotational behavior characterizing a superfluid. Both Raman-induced and Rashba spin-orbit couplings are considered. In the first case the moment of inertia takes the rigid value at the transition between the plane wave and the single minimum phase, while in the latter case the rigid value is achieved in the limit of isotropic Rashba coupling. A procedure to generate the rigid rotation of the fluid and to measure the moment of inertia is proposed. The quenching of the quantum of circulation $h/m$, caused by Raman induced spin-orbit coupling in a toroidal geometry, is also discussed.
\end{abstract}
\maketitle
\textit{Introduction.} Irrotationality is one of the most important features exhibited by superfluids \cite{Hosur,Hosur2}. It is at the origin of phenomena of fundamental relevance that have been confirmed experimentally both in superfluid helium and in ultracold atomic gases, like the quenching of the moment of inertia and the occurrence of quantized vortices. At small angular velocities the formation of quantized vortices is energetically inhibited and the moment of inertia of a superfluid, in the presence of isotropic confinement, vanishes at zero temperature as a consequence of the irrotationality of the velocity field. In Bose-Einstein condensates (BECs) the condition of irrotationality is usually associated with the phase $\phi$ of the order parameter, whose gradient fixes the superfluid velocity according to the relationship $v=(\hbar/m)
abla\phi$, where $m$ is the mass of atoms. However, even at $T=0$, the above irrotationality condition is violated in the presence of spin-orbit coupling, as we will discuss in the present Letter, with important consequences on the rotational behavior of the system.
Raman-induced spin-orbit coupled Bose gases, characterized by equal Rashba and Dresselhaus coupling, have been the object of systematic experimental and theoretical work in the recent years, following the pioneering experimental realization of \cite{Dresselhaus} (for recent reviews see, for example, \cite{Dresselhaus2,Dresselhaus3}). These configurations are characterized by the breaking of Galilean invariance which is responsible for important consequences like, for example, the breakdown of Landau's criterion for the critical velocity \cite{Landau,Landau1,Landau2}. Using Bogoliubov theory and sum rule arguments it has been recently shown \cite{Dresselhaus} that the normal (nonsuperfluid) density of these systems does not vanish at zero temperature, as happens in usual superfluids. In particular the normal density was calculated as a function of the Raman coupling produced by two lasers that couple two different hyperfine states, transferring momentum $\hbar k_0$. The effect is largest at the transition between the plane-wave phase and the single-minimum phase, where the motion of the fluid along the axis fixed by $k_0$, completely loses its superfluid nature even at $T=0$, despite the fact that the system is practically fully Bose-Einstein condensed \cite{Landau}.
The purpose of this letter is to investigate the rotational properties of spin-orbit coupled Bose-Einstein condensed gas employing a generalized hydrodynamic formalism, allowing for analytic solutions in the presence of harmonic trapping. In particular we prove the existence of a spin-orbit coupling of the system, which is the first one, the motion of the fluid along the axis of the system, and the second one, the motion of the superfluid along the axis of the system, which is the first one, the second one, the second one, the third one, the third one, the third one, the third one, the third one, the third one, the third one, the third one, the third one, the third one, the third one, the third one, the third one, the third one, the third one, the third one, the third one, the third one, the third one, the third one, the third one, the third one, the third one, the third one, the third one, the third one, the third one, the third one, the third one[SEP]
\importpackages{}
\graphicspath{ {./images/} }


\maketitle
\section{Introduction}
When the fermion zero modes localized on the surface or on the topological defects are studied in topological media, the investigation is mainly concentrated on the fully gapped topological media, such as topological insulators and superfluids/superconductors of the $^3$He-B type \cite{Fu,Qi,Fu2}. However, the gapless topological media may also have fermion zero modes with interesting properties, in particular they may have the dispersionless branch of spectrum with zero energy -- the flat band \cite{Fu2,Fu3}.
The dispersionless bands, where the energy vanishes in a finite region of the momentum space, have been discussed in different systems. Originally the flat band has been discussed in the fermionic condensate -- the Khodel state \cite{Khodel,Volovik,Volovik2,Volovik3}, and for fermion zero modes localized in the core of vortices in superfluid $^3$He-A \cite{Volovik4,Volovik5,Volovik6}. The flat band has also been discussed on the surface of the multi-layered graphene (see \cite{Volovik7,Volovik8} and references therein). In particle physics, the Fermi band (called the Fermi ball) appears in a 2+1 dimensional nonrelativistic quantum field theory which is dual to a gravitational theory in the anti-de Sitter background with a charged black hole \cite{Volovik8}.
Recently it was realized that the flat band can be topologically protected in gapless topological matter. It appears in the 3D systems which contain the nodal lines in the form of closed loops \cite{Volovik3} or in the form of spirals \cite{Volovik4}. In these systems the surface flat band emerges on the surface of topological matter. The boundary of the surface flat band is bounded by the projection of the nodal loop or nodal spiral onto the corresponding surface. Here we extend this bulk-surface correspondence to the bulk-vortex correspondence, which relates the flat band of fermion zero modes in the vortex core to the topology of the point nodes (Dirac or Fermi points) in the bulk 3D topological superfluids.
\section{Vortex-disgoration}
As generic example we consider topological defect in 3D spinless chiral superfluid/superconductor of the $^3$He-A type, which contains two Fermi points (Dirac points). Fermions in this chiral superfluid are described by Hamiltonian
\begin{eqnarray}
H&=&\tau_3\epsilon(p)+c\left(\tau_1p\cdot e_1+\tau_2p\cdot e_2\right),
onumber\\
\epsilon(p)&=&p^2-\frac{p^2}{F}2m\left(\frac{p^2}{F}\right),
\end{eqnarray}
where $\tau_{1,2,3}$ are Pauli matrices in the Bogoliubov-N\"{o}dinger-Lie-Fock (BK) operator $\tau_{1,2,3}$ and $\tau_{2,2,3}$ are Pauli matrices in the $^3$He-A operator $\tau_{1,2,3}$ and $\tau_{2,2,3}$ are Pauli matrices in the $^3$He-A operator $\tau_{1,2,3}$ and $\tau_{2,2,3}$ are Pauli matrices in the $^3$He-A operator $\tau_{1,2,3}$ and $\tau_{2,2,3}$ are Pauli matrices in the $^3$He-A operator $\tau_{1,2,3}$ and $\tau_{2,2,3}$ are Pauli matrices in the $^3$He-A operator $\tau_{1,2,3}$ and $\tau_{2,2,3}$ are Pauli matrices in the $^3$
\importpackages{}
\graphicspath{ {./images/} }


\begin{frontmatter}
\dochead{XXVIIIth International Conference on Ultrarelativistic Nucleus-Nucleus Collisions\\ (Quark Matter 2019)}
\title{Dijet Acoplanarity in CUJET3 as a Probe of the Nonperturbative Color Structure of QCD Perfect Fluids}
\author[label1,label2]{M. Gyulassy}
\author[label1]{P.M. Jacobs}
\author[label3]{J. Liao}
\author[label1,label2]{S. Shi}
\author[label1,label2]{X.N. Wang}
\author[label1]{F. Yuan}
\address[label1]{Nuclear Science Division, Lawrence Berkeley National Laboratory, Berkeley, CA 94720, USA}
\address[label2]{Institute of Particle Physics, Central China Normal University, Wuhan, China}
\address[label3]{Physics Department and Center for Exploration of Energy and Matter, Indiana University, 2401 N Milo B. Sampson Lane, Bloomington, IN 47408, USA}
\address[label4]{Dept. of Physics, McGill University, 3600 rue University, Montreal, QC H3A 2T8, Canada}
\begin{abstract}
Using the CUJET3=DGLV+VISHNU jet-medium interaction framework, we show that dijet azimuthal acoplanarity in high energy $A+A$ collisions is sensitive to possible non-perturbative enhancement of the jet transport coefficient, $\hat{q}(T,E)$, in the QCD crossover temperature $T\sim 150-300$ MeV range. With jet-medium couplings constrained by global RHIC\& LHC $\chi^2$ fits to nuclear modification data on $R_{AA}(p_T>20)$ GeV, we compare predictions of the medium induced dijet transverse momentum squared, $Q_s^2\sim \langle \hat{q}_L \rangle \sim \Delta\phi^2 E^2$, in two models of the temperature, $T$, and jet energy $E$ dependence of the jet medium transport coefficient, $\hat{q}(T,E)$. In one model, wQGP, the chromo degrees of freedom (dof) are approximated by a perturbative dielectric gas of quark and gluons dof. In the second model, sQGM, we consider a nonperturbative partially confined semi-Quark-Gluon-Monopole-Plasma with emergent color magnetic dof constrained by lattice QCD data. Unlike the slow variation of the scaled jet transport coefficient, $\hat{q}_{wQGP}/T^3$, the sQGM model $\hat{q}_{sQGM}/T^3$ features a sharp maximum in the QCD confinement crossover $T$ range. We show that the dijet path averaged medium induced azimuthal acoplanarity, $\Delta\phi^2$, in sQGM is robustly $\sim 2$ times larger than in perturbative wQGP. even though the radiative energy loss in both models is very similar as needed to fit the same $R_{AA}$ data. Future A+A dijet acoplanarity measurements constrained together with single jet $R_{AA}$ and $v_n$ measurements therefore appears to be a promising strategy to search for possible signatures of critical opalescence like phenomena in the QCD confinement temperature range.
\end{abstract}
\begin{keyword}
Quark Gluon Plasmas \sep Heavy Ion Collision \sep Jet Quenching \sep Dijet Acoplanarity
\end{keyword}
\end{frontmatter}
\section{Introduction and Conclusions}
Dijet relative azimuthal angle acoplanarity, $\Delta\phi^2=(\pi-\phi_1+\phi_2)^2=(Q_{vac}^2+Q_s^2)/E^2$, as a probe of Quark Gluon Plasmas (QGP) produced in high energy nuclear collisions, has a long history, see e.g. \cite{Baier:2010fb,Baier:2010fb,Baier:2010fb[SEP]
\importpackages{}
\graphicspath{ {./images/} }


\title{Rethinking Trajectory Evaluation for SLAM:\\ a Probabilistic, Continuous-Time Approach}
\author{Zichao Zhang, Davide Scaramuzza
\thanks{The authors are with the Robotics and Perception Group, Dep. of Informatics, University of Zurich, and Dep. of Neuroinformatics, University of Zurich and ETH Zurich, Switzerland--- http://rpg.ifi.uzh.ch.}
}
\maketitle
\thispagestyle{empty}
\pagestyle{empty}
\begin{abstract}
Despite the existence of different error metrics for trajectory evaluation in SLAM, their theoretical justifications and connections are rarely studied, and few methods handle temporal association properly. In this work, we propose to formulate the trajectory evaluation problem in a probabilistic, continuous-time framework. By modeling the groundtruth as random variables, the concepts of absolute and relative error are generalized to be likelihood. Moreover, the groundtruth is represented as a piecewise Gaussian Process in continuous-time. Within this framework, we are able to establish theoretical connections between relative and absolute error metrics and handle temporal association in a principled manner.
\end{abstract}
\section{INTRODUCTION}
Visual-(inertial) odometry (VO/VIO) and simultaneous localization and mapping (SLAM) are important building blocks in robotic systems, as they provide accurate state estimate for other tasks, such as control and planning. To benchmark such algorithms, the most used method is to evaluate the estimated trajectory (i.e., timestamped pose series) with respect to the groundtruth.
The central task for trajectory evaluation is to summarize certain metrics from the estimate and the groundtruth to indicate the performance. There are many established evaluation methods, most notably the absolute trajectory error (ATE) \cite{khan2016accuracy} and the relative error (RE) \cite{khan2017spatial}. While these methods are widely used in practice and can be indicative of the performance, there are still many open problems.
Specifically, in this paper, we are interested in the following:
\begin{enumerate}
    \item It is well known that different metrics reflect different properties of estimate \cite{khan2017spatial}. However, the connection between them is not clear. Indeed, it is observed in practice that relative and absolute errors are often highly correlated (e.g., \cite{khan2016accuracy}), but no theoretical proof has been proposed before.
    \item Almost all the existing methods assume perfect temporal correspondences or adopt a naive matching strategy. For example, to find the corresponding groundtruth of the estimate at time $t$, most tools simply use the closest groundtruth, which is only acceptable when the groundtruth is of sufficiently high temporal resolution. There is no principled method currently to take into consideration of the imperfect temporal association, which can in practice have an impact for low rate groundtruth providers or missing data.
\end{enumerate}
In this work, we proposes to formulate the trajectory evaluation in a probabilistic, continuous-time framework. First, instead of considering the groundtruth as deterministic values, we model it as random variables and thus generalize the concepts of absolute and relative error as likelihood. While this step seems trivial (e.g., sum of squared error is simply the likelihood from Gaussian uncertainties), this allows us to draw connection between relative and absolute error (Section \ref{sec:Gaussian}). Second, to reason about temporal association in a principled manner, we propose to use Gaussian Process (GP) to represent the groundtruth. As a probabilistic and continuous-time representation, GP reports uncertainty for any query time, which, for example, gives higher uncertainties for query times far away from the actual groundtruth samples. In this way, the effect of imperfect temporal association can be handled elegantly, which is not possible with many other continuous-time representations, such as polynomials.
It is well known that the trajectory evaluation problem is mainly complicated by the unobservable degrees-of-freedoms (DoFs) in the estimator \cite{khan2017spatial}, and thus is specific to different sensing modalities. For simplicity, throughout the paper, we present the framework for trajectory estimates with unknown rigid-body transformations (e.g., stereo or RGB-D sensors). However, our method can be adapted to other interesting setups (monocular, visual-inertial) in future work.
\subsection{Contributions and Outline}
The contributions of this work are summarized as follows:
\begin{enumerate}
    \item We propose to formulate the trajectory evaluation problem in a[SEP]
\importpackages{}
\graphicspath{ {./images/} }


\title{Binary Diversity for Non-Binary LDPC Codes over the Rayleigh Channel}
\author{\IEEEauthorblockN{Matteo Gorgoglione\IEEEauthorrefmark{1}, Valentin Savin\IEEEauthorrefmark{1}, David Declercq\IEEEauthorrefmark{4}}
\IEEEauthorblockA{\IEEEauthorrefmark{1}CEA-LETI, Minatec Campus, Grenoble, France \{matteo.gorgoglione, valentin.savin\}@cea.fr}
\IEEEauthorblockA{\IEEEauthorrefmark{4}ETIS, ENSEA Univ. Cergy-Pontoise/CNRS, Cergy-Pontoise, France, declercq@ensea.fr}
}
\maketitle
\begin{abstract}
In this paper we analyze the performance of several bit-interleaving strategies applied to Non-Binary Low-Density Parity-Check (LDPC) codes over the Rayleigh fading channel. The technique of bit-interleaving used over fading channel introduces diversity which could provide important gains in terms of frame error probability and detection. This paper demonstrates the importance of the way of implementing the bit-interleaving, and proposes a design of an optimized bit-interleaver inspired from the Progressive Edge Growth algorithm. This optimization algorithm depends on the topological structure of a given LDPC code and can also be applied to any degree distribution and code realization. In particular, we focus on non-binary LDPC codes based on graph with constant symbol-node connection $dv=2$. These regular $(2,dc)$-NB-LDPC codes demonstrate best performance, thanks to their large girths and improved decoding thresholds growing with the order of Finite Field. Simulations show excellent results of the proposed interleaving technique compared to the random interleaver as well as to the system without interleaver.
\end{abstract}
\IEEEpeerreviewmaketitle
\section{Introduction}
Since their rediscovery by MacKay \cite{MacKay} in 1996, Low-Density Parity-Check codes have attracted a lot of attention because they exhibit rates close to the Shannon capacity \cite{Shannon} for many transmission channels, despite their low decoding complexity. With the evolution of the technology, new families of LDPC codes defined on non-binary alphabets have been proposed and studied. They demonstrate better performance with respect to the binary case, especially for moderate code lengths \cite{LDPC} but at the expense of more complex decoding architectures.
Non-binary LDPC codes can be defined by considering a non-binary alphabet $A$, which for practical reasons is often considered to be endowed with a vector-space structure over the binary field $F_2=\{0,1\}$, and a semigroup $G$ acting on $A$. A non-binary code of length $N$ is hence defined as the set of solutions $s\in A$ of a linear system $H_{s_T}=0$, where $H$ is a matrix with coefficients in $G$, referred to as the parity-check matrix of the code.
LDPC codes are decoded using the belief propagation (BP) algorithm based on an iterative exchange of messages between nodes \cite{BP,BP2}.
The case of codes for which the underlying bipartite graph is ultra-sparse, in the sense that each symbol-node is connected to exactly $dv=2$ linear constraint-nodes, is of particular interest. First, very large girths can be obtained for Tanner graphs with $dv=2$, as demonstrated in \cite{LDPC2,LDPC3}. It has also been pointed out \cite{LDPC4,LDPC5} that when the size of the non-binary alphabet grows, best decoding thresholds are obtained for average density of edges closer and closer to $dv=2$.
Practically, for NB-LDPC codes defined over Galois fields $\mathbb{F}_q$ with $q\geq64$, best codes, both asymptotically and at finite lengths, are ultra-sparse codes. Despite those advantages, the ultra-sparse LDPC codes in $\mathbb{F}_q$ suffer from a serious drawback, as their minimum distance is limited and grows at best as $O(\log(N))$ \cite{LDPC5}. This limitation is however not critical when the desired error rate is above $10^{-5}$, which is the case of the wireless transmissions that we target in this paper[SEP]
\importpackages{}
\graphicspath{ {./images/} }


\title{Aharonov-Casher effect in Bi$_2$Se$_3$ square-ring interferometers}
\author{Fanming Qu}
\author{Fan Yang}
\author{Jun Chen}
\author{Jie Shen}
\author{Yue Ding}
\author{Jiangbo Lu}
\author{Yuanjun Song}
\author{Huaixin Yang}
\author{Guangtong Liu}
\author{Jie Fan}
\author{Yongqing Li}
\author{Zhongqing Ji}
\author{Changli Yang}
\author{Li Lu}
\email{luli@iphy.ac.cn}
\affiliation{Daniel Chee Tsui Laboratory, Beijing National Laboratory for Condensed Matter Physics, Institute of Physics, Chinese Academy of Sciences, Beijing 100190, People's Republic of China}
\date{\today}
\begin{abstract}
Electrical control of spin dynamics in Bi$_2$Se$_3$ was investigated in ring-type interferometers. Aharonov-Bohm and Altshuler-Aronov-Spivak resistance oscillations against magnetic field, and Aharorov-Casher resistance oscillations against gate voltage were observed in the presence of a Berry phase of $\pi$. A very large tunability of spin precession angle by gate voltage has been obtained, indicating that Bi$_2$Se$_3$-related materials with strong spin-orbit coupling are promising candidates for constructing novel spintronic devices.
\end{abstract}
\pacs{85.35.Ds, 85.75.-d, 71.70.Ej, 75.76.+j}
\maketitle
One important task of spintronics research is to explore electrical control of spin dynamics in solid-state devices via spin-orbit coupling (SOC) \cite{spintronics}. By tuning applied gate voltage one can rotate the spin of a moving electron with purely electrical means rather than traditionally with magnetic fields. The devices ever proposed and studied include the Datta-Das spin field effect transistors based on Rashba SOC \cite{Rashba, Rashba2}, the spin interferometers and filters \cite{Rashba2, Rashba3, Rashba4, Rashba5, Rashba6} via Aharonov-Casher (AC) effect \cite{AC}, and the quantum-dot spin qubits using electrical pulse to control the spin precession \cite{QT, QT2, QT3}, etc. Since the tunability of spin rotation by applied gate voltage relies on the strength of SOC, materials with stronger SOC would preferably be chosen to construct devices in this regard.
Topological insulators (TIs) \cite{TI1,TI2,TI3} are a new class of materials with strong SOC. The SOC in TIs is so strong that it leads to the formation of helical electron states at the surface/edge with inter-locked momentum and spin degrees of freedom. Many novel properties of TIs have been observed, including the formation of Dirac fermions \cite{TI4,TI5,TI6,TI7}, suppression of backscattering \cite{backscattering}, and the appearance of weak anti-localization (WAL) \cite{WAL1, WAL2, WAL3, WAL4, WAL5}. With such a novel electron system caused by strong SOC, electrical control of spin dynamics in TIs becomes a particularly interesting issue. Previously, Molenkamp group has studied the spin interference in a ring device made of HgTe/HgCdTe quantum well, a material which could be tuned into a two-dimensional (2D) TI, and reached a tunability higher than that in devices based on InAlAs/InGaAs two dimensional electron gas (2DEG) \cite{Rashba2, Rashba6}. Here we report our investigation on the electrical control of spin dynamics in a square-ring type of interferometer based on Bi$_2$Se$_3$, a material which could eventually be tuned into a 3D TI. The tunability of spin precession by gate voltage is found to be significantly higher than that reported before.
Bi$_2$Se$_3$ nanoplates were synthesized in a 2-inch horizontal tube furnace via a chemical vapor deposition method similar to literature \cite{HgTe}. Figure 1a and 1b are the scanning electron microscopy image and the high-resolution transmission electron microscopy image[SEP]
\importpackages{}
\graphicspath{ {./images/} }


\preprint{hep-ph/0304256}
\title{Rationalizing right-handed neutrinos}
\author{Graham D. Kribs}
\affiliation{Department of Physics, University of Wisconsin, Madison, WI 53706}
\email{kribs@physics.wisc.edu}
\begin{abstract}
A simple argument based on an SU(3) gauged horizontal symmetry is presented that connects the explanation for three generations of matter with the existence of a triplet of right-handed neutrinos. This rationale for right-handed neutrinos is analogous to, but completely independent of, grand unification or extra universal dimensions. A brief discussion of the supersymmetrized SU(3) model is also given, pointing out that certain problems in ordinary supersymmetric models such as fast proton decay via dimension-5 Planck-suppressed operators can be naturally solved.
\end{abstract}
\maketitle
One of the most remarkable features of the Standard Model (SM) is that matter fermions are chiral and yet all gauge~\cite{Weinberg:1978xd,Weinberg:1978xw} and gravitational~\cite{Weinberg:1978yv} anomalies vanish for each generation. A known but not often emphasized fact about the matter content is that, given one generation with unfixed hypercharges, anomaly cancellation determines the relative hypercharge assignment to be precisely what has been established by experiment~\cite{Weinberg:1978yv}. In other words, electric charge quantization is essentially automatic without grand unification. This fact, taken at face value, is circumstantial evidence against the existence of right-handed neutrinos. By definition a candidate for a right-handed neutrino is any fermion that is uncharged under all of the SM gauge symmetries. Yet, gauge symmetries are precisely the reason that each type of matter $(Q,u,d,L,e)$ is tied with the other matter fields together in a self-consistent, exclusive fashion. In addition, non-chiral matter allows a new mass scale unconnected to electroweak symmetry breaking that only further complicates our understanding of mass generation and mass hierarchies. Extensions of the SM with non-chiral matter, such as adding right-handed neutrinos, therefore appear to be contrary to all of the guiding wisdom gleaned from experiment, at least until recently. (Those who are still in doubt need only observe the emptor that the $\mu$ problem causes avatars of supersymmetry.)
Neutrino experiments~\cite{Aguilar:2001sx,Aguilar:2002sx,Aguilar:2002sx}, however, have firmly established that the neutrinos oscillate between each generation and thus they have mass. The largest mass of any one neutrino is constrained to be less than about 2 eV~\cite{Aguilar:2002sx}, and more likely their mass is one to a few orders of magnitude below this, depending on the generation. The mechanism of mass generation for neutrinos remains a mystery. If neutrinos acquire mass analogously to the SM matter fermions, preserving lepton number, then the particle content must be extended with at least two right-handed neutrinos $
u_{1,2}$. Ordinary Yukawa terms $L=\lambda_
u L H_
u c$ with tiny couplings $\lambda_
u \lesssim 10^{-12}$ suffice to explain the two undisputed mass differences found in neutrino oscillation experiments.
But, the global symmetry behind lepton number conservation is not expected to be exact. At dimension-5, the operator $H_{HL}/M$ violates lepton number by two units and leads to a tiny Majorana mass $v_2/M$ for left-handed neutrinos. This transmutes the neutrino mass hierarchy problem from explaining $\lambda_
u \lesssim 10^{-12}$ to instead explaining $v/M \lesssim 10^{-12}$. To embrace the dimension-5 neutrino mass explanation means the SM effective theory breaks down at $M \lesssim 10^{14}$ GeV. This is somewhat disconcerting since there are dimension-6 operators that violate lepton and baryon number, leading to a proton decay rate that is excluded by experiment unless $M \gtrsim 10^{16}$ GeV. Hence, while lepton number must be be violated at $M$ to explain neutrino masses, baryon number must be preserved to keep the proton stable. The simplest phenomenological explanation for lepton number violation without baryon number violation at the cutoff is the[SEP]
\importpackages{}
\graphicspath{ {./images/} }


\maketitle
\begin{abstract}
Let $M$ be a compact manifold of dimension at least 2. If $M$ admits a minimal homeomorphism then $M$ admits a minimal noninvertible map.
\end{abstract}
\section{Introduction}
Fathi and Herman \cite{Fathi} showed that every compact connected manifold which admits a smooth, locally free effective action of the circle group has a smooth diffeomorphism, isotopic to the identity, which is minimal, and so all odd dimensional spheres admit a minimal diffeomorphism (see also \cite{Fathi2}). Every manifold that admits a minimal flow admits also a minimal homeomorphism. The converse does not hold even in dimension 2, as manifested on the Klein bottle \cite{Klein}. Less is known about the existence of minimal noninvertible maps. The circle does not admit such maps, despite admitting minimal rotations \cite{Fathi}. Both the 2-torus and Klein bottle admit minimal noninvertible maps \cite{Fathi2,Fathi3}, but in dimension at least 3 it has not been known which manifolds admit minimal noninvertible maps. By application of a result of B\'{e}guin, Crovisier and Le Roux \cite{B\'{e}guin}, here we show in Theorem \ref{main} that any compact manifold of dimension at least 2 admits a minimal noninvertible map if it admits a minimal homeomorphism.
\section{Preliminaries}
A self map $f:X\rightarrow X$ of a compact metric space is said to be \emph{minimal} if for ever point $x\in X$ the forward orbit $\{f^n(x):n\in\mathbb{N}\}$ is dense in $X$.
\subsection{A Denjoy-type result}
The following result was obtained in \cite{B\'{e}guin}, p.304, as a corollary of a far-reaching generalization of the Denjoy-Rees technique. It will be crucial to the proof of Theorem \ref{main}.
\begin{theorem}[B\'{e}guin,Crovisier\&Le Roux]
Let $R$ be a homeomorphism on a compact manifold $M$, and $x$ a point of $M$ which is not periodic under $R$. Consider a compact subset $D$ of $M$ which can be written as the intersection of a strictly decreasing sequence $\{D^n:n\in\mathbb{N}\}$ of tamely embedded topological closed balls. Then there exist a homeomorphism $f:M\rightarrow M$ and a continuous onto map $\phi:M\rightarrow M$ such that $\phi\circ f=R\circ\phi$, and such that
\begin{itemize}
\item $\phi^{-1}(x)=D$;
\item $\phi^{-1}(y)$ is a single point if $y$ does not belong to the $R$-orbit of $x$.
\end{itemize}
\end{theorem}
\begin{remark}
It is implicitly mentioned in the proof of Proposition \ref{main}. in \cite{B\'{e}guin}, p.304, that for the homeomorphism $f$ in Theorem \ref{main} we have $\lim_{n\rightarrow\infty}\diam(f^n(D))=0$ (see also B3 and Proposition \ref{main} on p. 270 of \cite{B\'{e}guin}). The fact that if $R$ is minimal and $D$ has empty interior, then $f$ is minimal is explicitly stated in Remarks following Proposition \ref{main}. in \cite{B\'{e}guin}, p.304.
\end{remark}
\section{Main result}
Let $M$ be a compact manifold of dimension at least 2. If $M$ admits a minimal homeomorphism then $M$ admits a minimal noninvertible map.
\begin{theorem}[B\'{e}guin,Crovisier\&Le Roux]
Let $R$ be a minimal and empty interior. Then $f$ is minimal.
\end[SEP]